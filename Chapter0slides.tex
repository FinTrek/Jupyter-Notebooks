
% Default to the notebook output style

    


% Inherit from the specified cell style.




    
\documentclass[11pt]{article}

    
    
    \usepackage[T1]{fontenc}
    % Nicer default font (+ math font) than Computer Modern for most use cases
    \usepackage{mathpazo}

    % Basic figure setup, for now with no caption control since it's done
    % automatically by Pandoc (which extracts ![](path) syntax from Markdown).
    \usepackage{graphicx}
    % We will generate all images so they have a width \maxwidth. This means
    % that they will get their normal width if they fit onto the page, but
    % are scaled down if they would overflow the margins.
    \makeatletter
    \def\maxwidth{\ifdim\Gin@nat@width>\linewidth\linewidth
    \else\Gin@nat@width\fi}
    \makeatother
    \let\Oldincludegraphics\includegraphics
    % Set max figure width to be 80% of text width, for now hardcoded.
    \renewcommand{\includegraphics}[1]{\Oldincludegraphics[width=.8\maxwidth]{#1}}
    % Ensure that by default, figures have no caption (until we provide a
    % proper Figure object with a Caption API and a way to capture that
    % in the conversion process - todo).
    \usepackage{caption}
    \DeclareCaptionLabelFormat{nolabel}{}
    \captionsetup{labelformat=nolabel}

    \usepackage{adjustbox} % Used to constrain images to a maximum size 
    \usepackage{xcolor} % Allow colors to be defined
    \usepackage{enumerate} % Needed for markdown enumerations to work
    \usepackage{geometry} % Used to adjust the document margins
    \usepackage{amsmath} % Equations
    \usepackage{amssymb} % Equations
    \usepackage{textcomp} % defines textquotesingle
    % Hack from http://tex.stackexchange.com/a/47451/13684:
    \AtBeginDocument{%
        \def\PYZsq{\textquotesingle}% Upright quotes in Pygmentized code
    }
    \usepackage{upquote} % Upright quotes for verbatim code
    \usepackage{eurosym} % defines \euro
    \usepackage[mathletters]{ucs} % Extended unicode (utf-8) support
    \usepackage[utf8x]{inputenc} % Allow utf-8 characters in the tex document
    \usepackage{fancyvrb} % verbatim replacement that allows latex
    \usepackage{grffile} % extends the file name processing of package graphics 
                         % to support a larger range 
    % The hyperref package gives us a pdf with properly built
    % internal navigation ('pdf bookmarks' for the table of contents,
    % internal cross-reference links, web links for URLs, etc.)
    \usepackage{hyperref}
    \usepackage{longtable} % longtable support required by pandoc >1.10
    \usepackage{booktabs}  % table support for pandoc > 1.12.2
    \usepackage[inline]{enumitem} % IRkernel/repr support (it uses the enumerate* environment)
    \usepackage[normalem]{ulem} % ulem is needed to support strikethroughs (\sout)
                                % normalem makes italics be italics, not underlines
    

    
    
    % Colors for the hyperref package
    \definecolor{urlcolor}{rgb}{0,.145,.698}
    \definecolor{linkcolor}{rgb}{.71,0.21,0.01}
    \definecolor{citecolor}{rgb}{.12,.54,.11}

    % ANSI colors
    \definecolor{ansi-black}{HTML}{3E424D}
    \definecolor{ansi-black-intense}{HTML}{282C36}
    \definecolor{ansi-red}{HTML}{E75C58}
    \definecolor{ansi-red-intense}{HTML}{B22B31}
    \definecolor{ansi-green}{HTML}{00A250}
    \definecolor{ansi-green-intense}{HTML}{007427}
    \definecolor{ansi-yellow}{HTML}{DDB62B}
    \definecolor{ansi-yellow-intense}{HTML}{B27D12}
    \definecolor{ansi-blue}{HTML}{208FFB}
    \definecolor{ansi-blue-intense}{HTML}{0065CA}
    \definecolor{ansi-magenta}{HTML}{D160C4}
    \definecolor{ansi-magenta-intense}{HTML}{A03196}
    \definecolor{ansi-cyan}{HTML}{60C6C8}
    \definecolor{ansi-cyan-intense}{HTML}{258F8F}
    \definecolor{ansi-white}{HTML}{C5C1B4}
    \definecolor{ansi-white-intense}{HTML}{A1A6B2}

    % commands and environments needed by pandoc snippets
    % extracted from the output of `pandoc -s`
    \providecommand{\tightlist}{%
      \setlength{\itemsep}{0pt}\setlength{\parskip}{0pt}}
    \DefineVerbatimEnvironment{Highlighting}{Verbatim}{commandchars=\\\{\}}
    % Add ',fontsize=\small' for more characters per line
    \newenvironment{Shaded}{}{}
    \newcommand{\KeywordTok}[1]{\textcolor[rgb]{0.00,0.44,0.13}{\textbf{{#1}}}}
    \newcommand{\DataTypeTok}[1]{\textcolor[rgb]{0.56,0.13,0.00}{{#1}}}
    \newcommand{\DecValTok}[1]{\textcolor[rgb]{0.25,0.63,0.44}{{#1}}}
    \newcommand{\BaseNTok}[1]{\textcolor[rgb]{0.25,0.63,0.44}{{#1}}}
    \newcommand{\FloatTok}[1]{\textcolor[rgb]{0.25,0.63,0.44}{{#1}}}
    \newcommand{\CharTok}[1]{\textcolor[rgb]{0.25,0.44,0.63}{{#1}}}
    \newcommand{\StringTok}[1]{\textcolor[rgb]{0.25,0.44,0.63}{{#1}}}
    \newcommand{\CommentTok}[1]{\textcolor[rgb]{0.38,0.63,0.69}{\textit{{#1}}}}
    \newcommand{\OtherTok}[1]{\textcolor[rgb]{0.00,0.44,0.13}{{#1}}}
    \newcommand{\AlertTok}[1]{\textcolor[rgb]{1.00,0.00,0.00}{\textbf{{#1}}}}
    \newcommand{\FunctionTok}[1]{\textcolor[rgb]{0.02,0.16,0.49}{{#1}}}
    \newcommand{\RegionMarkerTok}[1]{{#1}}
    \newcommand{\ErrorTok}[1]{\textcolor[rgb]{1.00,0.00,0.00}{\textbf{{#1}}}}
    \newcommand{\NormalTok}[1]{{#1}}
    
    % Additional commands for more recent versions of Pandoc
    \newcommand{\ConstantTok}[1]{\textcolor[rgb]{0.53,0.00,0.00}{{#1}}}
    \newcommand{\SpecialCharTok}[1]{\textcolor[rgb]{0.25,0.44,0.63}{{#1}}}
    \newcommand{\VerbatimStringTok}[1]{\textcolor[rgb]{0.25,0.44,0.63}{{#1}}}
    \newcommand{\SpecialStringTok}[1]{\textcolor[rgb]{0.73,0.40,0.53}{{#1}}}
    \newcommand{\ImportTok}[1]{{#1}}
    \newcommand{\DocumentationTok}[1]{\textcolor[rgb]{0.73,0.13,0.13}{\textit{{#1}}}}
    \newcommand{\AnnotationTok}[1]{\textcolor[rgb]{0.38,0.63,0.69}{\textbf{\textit{{#1}}}}}
    \newcommand{\CommentVarTok}[1]{\textcolor[rgb]{0.38,0.63,0.69}{\textbf{\textit{{#1}}}}}
    \newcommand{\VariableTok}[1]{\textcolor[rgb]{0.10,0.09,0.49}{{#1}}}
    \newcommand{\ControlFlowTok}[1]{\textcolor[rgb]{0.00,0.44,0.13}{\textbf{{#1}}}}
    \newcommand{\OperatorTok}[1]{\textcolor[rgb]{0.40,0.40,0.40}{{#1}}}
    \newcommand{\BuiltInTok}[1]{{#1}}
    \newcommand{\ExtensionTok}[1]{{#1}}
    \newcommand{\PreprocessorTok}[1]{\textcolor[rgb]{0.74,0.48,0.00}{{#1}}}
    \newcommand{\AttributeTok}[1]{\textcolor[rgb]{0.49,0.56,0.16}{{#1}}}
    \newcommand{\InformationTok}[1]{\textcolor[rgb]{0.38,0.63,0.69}{\textbf{\textit{{#1}}}}}
    \newcommand{\WarningTok}[1]{\textcolor[rgb]{0.38,0.63,0.69}{\textbf{\textit{{#1}}}}}
    
    
    % Define a nice break command that doesn't care if a line doesn't already
    % exist.
    \def\br{\hspace*{\fill} \\* }
    % Math Jax compatability definitions
    \def\gt{>}
    \def\lt{<}
    % Document parameters
    \title{Chapter0slides}
    
    
    

    % Pygments definitions
    
\makeatletter
\def\PY@reset{\let\PY@it=\relax \let\PY@bf=\relax%
    \let\PY@ul=\relax \let\PY@tc=\relax%
    \let\PY@bc=\relax \let\PY@ff=\relax}
\def\PY@tok#1{\csname PY@tok@#1\endcsname}
\def\PY@toks#1+{\ifx\relax#1\empty\else%
    \PY@tok{#1}\expandafter\PY@toks\fi}
\def\PY@do#1{\PY@bc{\PY@tc{\PY@ul{%
    \PY@it{\PY@bf{\PY@ff{#1}}}}}}}
\def\PY#1#2{\PY@reset\PY@toks#1+\relax+\PY@do{#2}}

\expandafter\def\csname PY@tok@w\endcsname{\def\PY@tc##1{\textcolor[rgb]{0.73,0.73,0.73}{##1}}}
\expandafter\def\csname PY@tok@c\endcsname{\let\PY@it=\textit\def\PY@tc##1{\textcolor[rgb]{0.25,0.50,0.50}{##1}}}
\expandafter\def\csname PY@tok@cp\endcsname{\def\PY@tc##1{\textcolor[rgb]{0.74,0.48,0.00}{##1}}}
\expandafter\def\csname PY@tok@k\endcsname{\let\PY@bf=\textbf\def\PY@tc##1{\textcolor[rgb]{0.00,0.50,0.00}{##1}}}
\expandafter\def\csname PY@tok@kp\endcsname{\def\PY@tc##1{\textcolor[rgb]{0.00,0.50,0.00}{##1}}}
\expandafter\def\csname PY@tok@kt\endcsname{\def\PY@tc##1{\textcolor[rgb]{0.69,0.00,0.25}{##1}}}
\expandafter\def\csname PY@tok@o\endcsname{\def\PY@tc##1{\textcolor[rgb]{0.40,0.40,0.40}{##1}}}
\expandafter\def\csname PY@tok@ow\endcsname{\let\PY@bf=\textbf\def\PY@tc##1{\textcolor[rgb]{0.67,0.13,1.00}{##1}}}
\expandafter\def\csname PY@tok@nb\endcsname{\def\PY@tc##1{\textcolor[rgb]{0.00,0.50,0.00}{##1}}}
\expandafter\def\csname PY@tok@nf\endcsname{\def\PY@tc##1{\textcolor[rgb]{0.00,0.00,1.00}{##1}}}
\expandafter\def\csname PY@tok@nc\endcsname{\let\PY@bf=\textbf\def\PY@tc##1{\textcolor[rgb]{0.00,0.00,1.00}{##1}}}
\expandafter\def\csname PY@tok@nn\endcsname{\let\PY@bf=\textbf\def\PY@tc##1{\textcolor[rgb]{0.00,0.00,1.00}{##1}}}
\expandafter\def\csname PY@tok@ne\endcsname{\let\PY@bf=\textbf\def\PY@tc##1{\textcolor[rgb]{0.82,0.25,0.23}{##1}}}
\expandafter\def\csname PY@tok@nv\endcsname{\def\PY@tc##1{\textcolor[rgb]{0.10,0.09,0.49}{##1}}}
\expandafter\def\csname PY@tok@no\endcsname{\def\PY@tc##1{\textcolor[rgb]{0.53,0.00,0.00}{##1}}}
\expandafter\def\csname PY@tok@nl\endcsname{\def\PY@tc##1{\textcolor[rgb]{0.63,0.63,0.00}{##1}}}
\expandafter\def\csname PY@tok@ni\endcsname{\let\PY@bf=\textbf\def\PY@tc##1{\textcolor[rgb]{0.60,0.60,0.60}{##1}}}
\expandafter\def\csname PY@tok@na\endcsname{\def\PY@tc##1{\textcolor[rgb]{0.49,0.56,0.16}{##1}}}
\expandafter\def\csname PY@tok@nt\endcsname{\let\PY@bf=\textbf\def\PY@tc##1{\textcolor[rgb]{0.00,0.50,0.00}{##1}}}
\expandafter\def\csname PY@tok@nd\endcsname{\def\PY@tc##1{\textcolor[rgb]{0.67,0.13,1.00}{##1}}}
\expandafter\def\csname PY@tok@s\endcsname{\def\PY@tc##1{\textcolor[rgb]{0.73,0.13,0.13}{##1}}}
\expandafter\def\csname PY@tok@sd\endcsname{\let\PY@it=\textit\def\PY@tc##1{\textcolor[rgb]{0.73,0.13,0.13}{##1}}}
\expandafter\def\csname PY@tok@si\endcsname{\let\PY@bf=\textbf\def\PY@tc##1{\textcolor[rgb]{0.73,0.40,0.53}{##1}}}
\expandafter\def\csname PY@tok@se\endcsname{\let\PY@bf=\textbf\def\PY@tc##1{\textcolor[rgb]{0.73,0.40,0.13}{##1}}}
\expandafter\def\csname PY@tok@sr\endcsname{\def\PY@tc##1{\textcolor[rgb]{0.73,0.40,0.53}{##1}}}
\expandafter\def\csname PY@tok@ss\endcsname{\def\PY@tc##1{\textcolor[rgb]{0.10,0.09,0.49}{##1}}}
\expandafter\def\csname PY@tok@sx\endcsname{\def\PY@tc##1{\textcolor[rgb]{0.00,0.50,0.00}{##1}}}
\expandafter\def\csname PY@tok@m\endcsname{\def\PY@tc##1{\textcolor[rgb]{0.40,0.40,0.40}{##1}}}
\expandafter\def\csname PY@tok@gh\endcsname{\let\PY@bf=\textbf\def\PY@tc##1{\textcolor[rgb]{0.00,0.00,0.50}{##1}}}
\expandafter\def\csname PY@tok@gu\endcsname{\let\PY@bf=\textbf\def\PY@tc##1{\textcolor[rgb]{0.50,0.00,0.50}{##1}}}
\expandafter\def\csname PY@tok@gd\endcsname{\def\PY@tc##1{\textcolor[rgb]{0.63,0.00,0.00}{##1}}}
\expandafter\def\csname PY@tok@gi\endcsname{\def\PY@tc##1{\textcolor[rgb]{0.00,0.63,0.00}{##1}}}
\expandafter\def\csname PY@tok@gr\endcsname{\def\PY@tc##1{\textcolor[rgb]{1.00,0.00,0.00}{##1}}}
\expandafter\def\csname PY@tok@ge\endcsname{\let\PY@it=\textit}
\expandafter\def\csname PY@tok@gs\endcsname{\let\PY@bf=\textbf}
\expandafter\def\csname PY@tok@gp\endcsname{\let\PY@bf=\textbf\def\PY@tc##1{\textcolor[rgb]{0.00,0.00,0.50}{##1}}}
\expandafter\def\csname PY@tok@go\endcsname{\def\PY@tc##1{\textcolor[rgb]{0.53,0.53,0.53}{##1}}}
\expandafter\def\csname PY@tok@gt\endcsname{\def\PY@tc##1{\textcolor[rgb]{0.00,0.27,0.87}{##1}}}
\expandafter\def\csname PY@tok@err\endcsname{\def\PY@bc##1{\setlength{\fboxsep}{0pt}\fcolorbox[rgb]{1.00,0.00,0.00}{1,1,1}{\strut ##1}}}
\expandafter\def\csname PY@tok@kc\endcsname{\let\PY@bf=\textbf\def\PY@tc##1{\textcolor[rgb]{0.00,0.50,0.00}{##1}}}
\expandafter\def\csname PY@tok@kd\endcsname{\let\PY@bf=\textbf\def\PY@tc##1{\textcolor[rgb]{0.00,0.50,0.00}{##1}}}
\expandafter\def\csname PY@tok@kn\endcsname{\let\PY@bf=\textbf\def\PY@tc##1{\textcolor[rgb]{0.00,0.50,0.00}{##1}}}
\expandafter\def\csname PY@tok@kr\endcsname{\let\PY@bf=\textbf\def\PY@tc##1{\textcolor[rgb]{0.00,0.50,0.00}{##1}}}
\expandafter\def\csname PY@tok@bp\endcsname{\def\PY@tc##1{\textcolor[rgb]{0.00,0.50,0.00}{##1}}}
\expandafter\def\csname PY@tok@fm\endcsname{\def\PY@tc##1{\textcolor[rgb]{0.00,0.00,1.00}{##1}}}
\expandafter\def\csname PY@tok@vc\endcsname{\def\PY@tc##1{\textcolor[rgb]{0.10,0.09,0.49}{##1}}}
\expandafter\def\csname PY@tok@vg\endcsname{\def\PY@tc##1{\textcolor[rgb]{0.10,0.09,0.49}{##1}}}
\expandafter\def\csname PY@tok@vi\endcsname{\def\PY@tc##1{\textcolor[rgb]{0.10,0.09,0.49}{##1}}}
\expandafter\def\csname PY@tok@vm\endcsname{\def\PY@tc##1{\textcolor[rgb]{0.10,0.09,0.49}{##1}}}
\expandafter\def\csname PY@tok@sa\endcsname{\def\PY@tc##1{\textcolor[rgb]{0.73,0.13,0.13}{##1}}}
\expandafter\def\csname PY@tok@sb\endcsname{\def\PY@tc##1{\textcolor[rgb]{0.73,0.13,0.13}{##1}}}
\expandafter\def\csname PY@tok@sc\endcsname{\def\PY@tc##1{\textcolor[rgb]{0.73,0.13,0.13}{##1}}}
\expandafter\def\csname PY@tok@dl\endcsname{\def\PY@tc##1{\textcolor[rgb]{0.73,0.13,0.13}{##1}}}
\expandafter\def\csname PY@tok@s2\endcsname{\def\PY@tc##1{\textcolor[rgb]{0.73,0.13,0.13}{##1}}}
\expandafter\def\csname PY@tok@sh\endcsname{\def\PY@tc##1{\textcolor[rgb]{0.73,0.13,0.13}{##1}}}
\expandafter\def\csname PY@tok@s1\endcsname{\def\PY@tc##1{\textcolor[rgb]{0.73,0.13,0.13}{##1}}}
\expandafter\def\csname PY@tok@mb\endcsname{\def\PY@tc##1{\textcolor[rgb]{0.40,0.40,0.40}{##1}}}
\expandafter\def\csname PY@tok@mf\endcsname{\def\PY@tc##1{\textcolor[rgb]{0.40,0.40,0.40}{##1}}}
\expandafter\def\csname PY@tok@mh\endcsname{\def\PY@tc##1{\textcolor[rgb]{0.40,0.40,0.40}{##1}}}
\expandafter\def\csname PY@tok@mi\endcsname{\def\PY@tc##1{\textcolor[rgb]{0.40,0.40,0.40}{##1}}}
\expandafter\def\csname PY@tok@il\endcsname{\def\PY@tc##1{\textcolor[rgb]{0.40,0.40,0.40}{##1}}}
\expandafter\def\csname PY@tok@mo\endcsname{\def\PY@tc##1{\textcolor[rgb]{0.40,0.40,0.40}{##1}}}
\expandafter\def\csname PY@tok@ch\endcsname{\let\PY@it=\textit\def\PY@tc##1{\textcolor[rgb]{0.25,0.50,0.50}{##1}}}
\expandafter\def\csname PY@tok@cm\endcsname{\let\PY@it=\textit\def\PY@tc##1{\textcolor[rgb]{0.25,0.50,0.50}{##1}}}
\expandafter\def\csname PY@tok@cpf\endcsname{\let\PY@it=\textit\def\PY@tc##1{\textcolor[rgb]{0.25,0.50,0.50}{##1}}}
\expandafter\def\csname PY@tok@c1\endcsname{\let\PY@it=\textit\def\PY@tc##1{\textcolor[rgb]{0.25,0.50,0.50}{##1}}}
\expandafter\def\csname PY@tok@cs\endcsname{\let\PY@it=\textit\def\PY@tc##1{\textcolor[rgb]{0.25,0.50,0.50}{##1}}}

\def\PYZbs{\char`\\}
\def\PYZus{\char`\_}
\def\PYZob{\char`\{}
\def\PYZcb{\char`\}}
\def\PYZca{\char`\^}
\def\PYZam{\char`\&}
\def\PYZlt{\char`\<}
\def\PYZgt{\char`\>}
\def\PYZsh{\char`\#}
\def\PYZpc{\char`\%}
\def\PYZdl{\char`\$}
\def\PYZhy{\char`\-}
\def\PYZsq{\char`\'}
\def\PYZdq{\char`\"}
\def\PYZti{\char`\~}
% for compatibility with earlier versions
\def\PYZat{@}
\def\PYZlb{[}
\def\PYZrb{]}
\makeatother


    % Exact colors from NB
    \definecolor{incolor}{rgb}{0.0, 0.0, 0.5}
    \definecolor{outcolor}{rgb}{0.545, 0.0, 0.0}



    
    % Prevent overflowing lines due to hard-to-break entities
    \sloppy 
    % Setup hyperref package
    \hypersetup{
      breaklinks=true,  % so long urls are correctly broken across lines
      colorlinks=true,
      urlcolor=urlcolor,
      linkcolor=linkcolor,
      citecolor=citecolor,
      }
    % Slightly bigger margins than the latex defaults
    
    \geometry{verbose,tmargin=1in,bmargin=1in,lmargin=1in,rmargin=1in}
    
    

    \begin{document}
    
    
    \maketitle
    
    

    
    \begin{itemize}
\item ~
  \section{Chapter 0: Administrative}\label{chapter-0-administrative}

  \begin{itemize}
  \item ~
    \subsection{First class is January 8, 2019 from 10:30AM to 12 noon
    in Baxter
    128.}\label{first-class-is-january-8-2019-from-1030am-to-12-noon-in-baxter-128.}
  \item ~
    \subsection{Professor is Kenneth Winston (kwinston@caltech.edu). My
    office is 236 Baxter. I am there after every class, and lots of
    other times. Email if you can't find me and want to set up a
    meeting.}\label{professor-is-kenneth-winston-kwinstoncaltech.edu.-my-office-is-236-baxter.-i-am-there-after-every-class-and-lots-of-other-times.-email-if-you-cant-find-me-and-want-to-set-up-a-meeting.}
  \item ~
    \subsection{This is the ninth year I have taught this class. This
    Jupyter notebook is the
    textbook.}\label{this-is-the-ninth-year-i-have-taught-this-class.-this-jupyter-notebook-is-the-textbook.}
  \item ~
    \subsection{I generally post the next week's lectures on Sunday
    nights.}\label{i-generally-post-the-next-weeks-lectures-on-sunday-nights.}
  \item ~
    \subsection{This class requires a fair amount of time. You should
    plan to spend at least nine hours a week on the class between
    lectures, homework, tests, and general
    studying.}\label{this-class-requires-a-fair-amount-of-time.-you-should-plan-to-spend-at-least-nine-hours-a-week-on-the-class-between-lectures-homework-tests-and-general-studying.}
  \end{itemize}
\end{itemize}

Copyright © Kenneth Winston 2019

    \begin{itemize}
\item ~
  \section{\texorpdfstring{{0.1 Components of the
  class}}{0.1 Components of the class}}\label{components-of-the-class}

  \begin{itemize}
  \item ~
    \subsection{\texorpdfstring{{0.1.1 Class attendance and
    participation}}{0.1.1 Class attendance and participation}}\label{class-attendance-and-participation}

    \begin{itemize}
    \item ~
      \subsubsection{There will be twenty 1.5 hour
      lectures}\label{there-will-be-twenty-1.5-hour-lectures}
    \item ~
      \subsubsection{People learn more when they participate in class,
      especially when they ask
      questions}\label{people-learn-more-when-they-participate-in-class-especially-when-they-ask-questions}
    \item ~
      \subsubsection{The cases where}\label{the-cases-where}

      \begin{quote}
      \begin{enumerate}
      \def\labelenumi{(\alph{enumi})}
      \tightlist
      \item
        you are sitting in class thinking ``I'm the only idiot who
        doesn't understand this;'' and
      \end{enumerate}
      \end{quote}

      \begin{enumerate}
      \def\labelenumi{(\alph{enumi})}
      \setcounter{enumi}{1}
      \tightlist
      \item
        you actually are the only idiot who doesn't understand this
      \end{enumerate}
    \item ~
      \subsubsection{form a set of measure
      zero.}\label{form-a-set-of-measure-zero.}
    \item ~
      \subsubsection{So ask questions. Probably the professor is the
      idiot for not explaining it properly, and your classmates will
      thank you for forcing a
      clarification.}\label{so-ask-questions.-probably-the-professor-is-the-idiot-for-not-explaining-it-properly-and-your-classmates-will-thank-you-for-forcing-a-clarification.}
    \item ~
      \subsubsection{If you take the class pass/fail, you must attend
      the entire lecture for sixteen out of the twenty lectures to pass.
      ("Attend" means you are paying attention to the class, not swiping
      right on your
      phone.)}\label{if-you-take-the-class-passfail-you-must-attend-the-entire-lecture-for-sixteen-out-of-the-twenty-lectures-to-pass.-attend-means-you-are-paying-attention-to-the-class-not-swiping-right-on-your-phone.}
    \item ~
      \subsubsection{If you take the class for a grade, you do not have
      to attend the lectures. However the maximum you can get without
      attending at least sixteen lectures is
      A-.}\label{if-you-take-the-class-for-a-grade-you-do-not-have-to-attend-the-lectures.-however-the-maximum-you-can-get-without-attending-at-least-sixteen-lectures-is-a-.}
    \end{itemize}
  \end{itemize}
\end{itemize}

    \begin{itemize}
\item ~
  \section{\texorpdfstring{{0.1 Components, cont'd
  (1)}}{0.1 Components, cont'd (1)}}\label{components-contd-1}

  \begin{itemize}
  \item ~
    \subsection{0.1.2 Homework, 30\%.}\label{homework-30.}

    \begin{itemize}
    \item ~
      \subsubsection{Homework assignments are at the end of each
      Thursday lecture except Lecture 10 and Lecture
      20.}\label{homework-assignments-are-at-the-end-of-each-thursday-lecture-except-lecture-10-and-lecture-20.}
    \item ~
      \subsubsection{Responses should be uploaded to Moodle before the
      beginning of the next Thursday
      class.}\label{responses-should-be-uploaded-to-moodle-before-the-beginning-of-the-next-thursday-class.}
    \item ~
      \subsubsection{Original electronic documents are preferred (PDF,
      Word) but if you need to, you can write on paper, scan to PDF or
      JPG; and
      upload.}\label{original-electronic-documents-are-preferred-pdf-word-but-if-you-need-to-you-can-write-on-paper-scan-to-pdf-or-jpg-and-upload.}
    \item ~
      \subsubsection{Since we may discuss the homework in class, no
      credit will be given for homework received after the start of the
      class where it is due. The two worst homeworks will be thrown out;
      the remaining ones will be averaged to form 30\% of the
      grade.}\label{since-we-may-discuss-the-homework-in-class-no-credit-will-be-given-for-homework-received-after-the-start-of-the-class-where-it-is-due.-the-two-worst-homeworks-will-be-thrown-out-the-remaining-ones-will-be-averaged-to-form-30-of-the-grade.}
    \item ~
      \subsubsection{You may collaborate on homework with other members
      of this year's class, but you cannot collaborate with any other
      person, nor can you use materials or information provided by
      people who have taken the class in previous years. Other than
      that, you can use any resource you want while doing the
      homework.}\label{you-may-collaborate-on-homework-with-other-members-of-this-years-class-but-you-cannot-collaborate-with-any-other-person-nor-can-you-use-materials-or-information-provided-by-people-who-have-taken-the-class-in-previous-years.-other-than-that-you-can-use-any-resource-you-want-while-doing-the-homework.}
    \item ~
      \subsubsection{\texorpdfstring{You get 10 additional homework
      points (/100) for each major error you discover in my lecture
      slides or Python code. I am the sole judge of what constitutes a
      major error, but for example if I say ``\emph{Tesla, Inc. is a
      company that operates coal mines,}'' that's a major error. Eight
      previous years of students have pored over this material, so good
      luck. But I make a lot of changes each year, so that's a rich
      source of new errors = bonus homework
      points.}{You get 10 additional homework points (/100) for each major error you discover in my lecture slides or Python code. I am the sole judge of what constitutes a major error, but for example if I say ``Tesla, Inc. is a company that operates coal mines,'' that's a major error. Eight previous years of students have pored over this material, so good luck. But I make a lot of changes each year, so that's a rich source of new errors = bonus homework points.}}\label{you-get-10-additional-homework-points-100-for-each-major-error-you-discover-in-my-lecture-slides-or-python-code.-i-am-the-sole-judge-of-what-constitutes-a-major-error-but-for-example-if-i-say-tesla-inc.-is-a-company-that-operates-coal-mines-thats-a-major-error.-eight-previous-years-of-students-have-pored-over-this-material-so-good-luck.-but-i-make-a-lot-of-changes-each-year-so-thats-a-rich-source-of-new-errors-bonus-homework-points.}
    \end{itemize}
  \end{itemize}
\end{itemize}

    \begin{itemize}
\item ~
  \section{\texorpdfstring{{0.1 Components, cont'd
  (2)}}{0.1 Components, cont'd (2)}}\label{components-contd-2}

  \begin{itemize}
  \item ~
    \subsection{\texorpdfstring{{0.1.3 Midterm,
    30\%.}}{0.1.3 Midterm, 30\%.}}\label{midterm-30.}

    \begin{itemize}
    \item ~
      \subsubsection{A midterm will be given after Lecture
      10.}\label{a-midterm-will-be-given-after-lecture-10.}
    \item ~
      \subsubsection{The midterm is take-home and is due just before
      Lecture
      12.}\label{the-midterm-is-take-home-and-is-due-just-before-lecture-12.}
    \item ~
      \subsubsection{You may NOT collaborate on the midterm (or final)
      with any other person, nor can you use materials or information
      provided by people who have taken the class in previous years.
      Other than that, you can use any resource you want while doing the
      midterm (or
      final).}\label{you-may-not-collaborate-on-the-midterm-or-final-with-any-other-person-nor-can-you-use-materials-or-information-provided-by-people-who-have-taken-the-class-in-previous-years.-other-than-that-you-can-use-any-resource-you-want-while-doing-the-midterm-or-final.}
    \end{itemize}
  \item ~
    \section{\texorpdfstring{{0.1.4 Final,
    40\%.}}{0.1.4 Final, 40\%.}}\label{final-40.}

    \begin{itemize}
    \item ~
      \subsubsection{A final will be given after Lecture 20. The final
      is also take-home and will cover the entire class including
      Lectures 1-10. If you are taking the class pass/fail, you must
      score at least 65\% on the
      final.}\label{a-final-will-be-given-after-lecture-20.-the-final-is-also-take-home-and-will-cover-the-entire-class-including-lectures-1-10.-if-you-are-taking-the-class-passfail-you-must-score-at-least-65-on-the-final.}
    \end{itemize}
  \end{itemize}
\end{itemize}

    \begin{itemize}
\item ~
  \section{0.2 In summary, a grade will be formed
  from:}\label{in-summary-a-grade-will-be-formed-from}
\item ~
  \subsection{30\% homework}\label{homework}
\item ~
  \subsection{30\% midterm}\label{midterm}
\item ~
  \subsection{40\% final.}\label{final.}
\item ~
  \section{If you are taking the class pass/fail, you will fail if you
  did not attend sixteen out of twenty lectures even if your score is
  otherwise passing. You will also fail if you do not get at least 65\%
  on the final even if your score is otherwise
  passing.}\label{if-you-are-taking-the-class-passfail-you-will-fail-if-you-did-not-attend-sixteen-out-of-twenty-lectures-even-if-your-score-is-otherwise-passing.-you-will-also-fail-if-you-do-not-get-at-least-65-on-the-final-even-if-your-score-is-otherwise-passing.}
\item ~
  \section{I grade a little bit on a curve so if everyone is doing great
  the bar goes up and if everyone is doing badly the bar goes down. But
  generally consider the
  mapping}\label{i-grade-a-little-bit-on-a-curve-so-if-everyone-is-doing-great-the-bar-goes-up-and-if-everyone-is-doing-badly-the-bar-goes-down.-but-generally-consider-the-mapping}
\item ~
  \subsection{f(9)=A;}\label{f9a}
\item ~
  \subsection{f(8)=B;}\label{f8b}
\item ~
  \subsection{f(7)=C,}\label{f7c}
\item ~
  \subsection{f(6)=D.}\label{f6d.}
\item ~
  \subsection{g(x)=+ if
  10\textgreater{}x\textgreater{}6.667;}\label{gx-if-10x6.667}
\item ~
  \subsection{g(x)=blank if 6.667\textgreater{}x\textgreater{}3.333;
  and}\label{gxblank-if-6.667x3.333-and}
\item ~
  \subsection{g(x)=- if
  3.333\textgreater{}x\textgreater{}0.}\label{gx--if-3.333x0.}
\item ~
  \section{Then your letter grade (except for curve adjustments) is
  f({[}n/10{]}) concatenated with g(n mod 10) where n is your numerical
  grade. Above 100 (which several people have gotten) is of course A+.
  Below 63.333 is probably failing although I might move it a bit
  depending on the rest of the
  class.}\label{then-your-letter-grade-except-for-curve-adjustments-is-fn10-concatenated-with-gn-mod-10-where-n-is-your-numerical-grade.-above-100-which-several-people-have-gotten-is-of-course-a.-below-63.333-is-probably-failing-although-i-might-move-it-a-bit-depending-on-the-rest-of-the-class.}
\item ~
  \section{Use computers only for calculation. If you are taking an
  integral or a derivative, solve it yourself: don't just say ``Wolfram
  Alpha says the answer is\ldots{}'' And if you do use a computer for
  calculation, show enough intermediate steps to prove that you know how
  the theory works. I prefer well-commented Python code for assignments
  that require a
  computer.}\label{use-computers-only-for-calculation.-if-you-are-taking-an-integral-or-a-derivative-solve-it-yourself-dont-just-say-wolfram-alpha-says-the-answer-is-and-if-you-do-use-a-computer-for-calculation-show-enough-intermediate-steps-to-prove-that-you-know-how-the-theory-works.-i-prefer-well-commented-python-code-for-assignments-that-require-a-computer.}
\end{itemize}

    \begin{itemize}
\item ~
  \section{0.3 About me; BEM 111
  background}\label{about-me-bem-111-background}
\item ~
  \subsection{I grew up near Boston, Massachusetts. I went to Caltech
  and got a BS and MS in mathematics. I got a PhD in mathematics
  (combinatorics) from
  M.I.T.}\label{i-grew-up-near-boston-massachusetts.-i-went-to-caltech-and-got-a-bs-and-ms-in-mathematics.-i-got-a-phd-in-mathematics-combinatorics-from-m.i.t.}
\item ~
  \subsection{After getting my PhD, I briefly taught mathematics at
  Rutgers University. I then got a job as a portfolio manager for a
  small firm in New York City. I managed portfolios (equity and
  derivatives) using quantitative methods. We will talk about some of
  the theories I and others used in this class. Eventually I moved into
  risk management, which I found more interesting because it is less
  narrowly focused than portfolio
  management.}\label{after-getting-my-phd-i-briefly-taught-mathematics-at-rutgers-university.-i-then-got-a-job-as-a-portfolio-manager-for-a-small-firm-in-new-york-city.-i-managed-portfolios-equity-and-derivatives-using-quantitative-methods.-we-will-talk-about-some-of-the-theories-i-and-others-used-in-this-class.-eventually-i-moved-into-risk-management-which-i-found-more-interesting-because-it-is-less-narrowly-focused-than-portfolio-management.}
\item ~
  \subsection{I was the chief risk officer at Morgan Stanley Investment
  Management in New York, and also worked in ``sell side''
  (broker-dealer) risk at Morgan Stanley. I came back to Pasadena in
  2008 to work as the chief risk officer at Western Asset Management,
  which (1) manages about \$410 billion; and (2) is right down Colorado
  Blvd here in
  Pasadena.}\label{i-was-the-chief-risk-officer-at-morgan-stanley-investment-management-in-new-york-and-also-worked-in-sell-side-broker-dealer-risk-at-morgan-stanley.-i-came-back-to-pasadena-in-2008-to-work-as-the-chief-risk-officer-at-western-asset-management-which-1-manages-about-410-billion-and-2-is-right-down-colorado-blvd-here-in-pasadena.}
\item ~
  \subsection{While I was at Morgan Stanley, I was an adjunct professor
  at the Courant Institute of Mathematical Sciences at New York
  University. I taught a class in Capital Markets and Portfolio Theory
  there; that class was a bit similar to BEM 104 at
  Caltech.}\label{while-i-was-at-morgan-stanley-i-was-an-adjunct-professor-at-the-courant-institute-of-mathematical-sciences-at-new-york-university.-i-taught-a-class-in-capital-markets-and-portfolio-theory-there-that-class-was-a-bit-similar-to-bem-104-at-caltech.}
\item ~
  \subsection{Despite the fact that we are in the humanities \& social
  sciences division, this is basically an applied mathematics class.
  Reasonably trained Caltech students who meet the prerequisites should
  be able to handle it, but it will definitely be more quantitative than
  qualitative. Today's class is by far the least quantitative, but it
  will give you the flavor of what we'll
  do.}\label{despite-the-fact-that-we-are-in-the-humanities-social-sciences-division-this-is-basically-an-applied-mathematics-class.-reasonably-trained-caltech-students-who-meet-the-prerequisites-should-be-able-to-handle-it-but-it-will-definitely-be-more-quantitative-than-qualitative.-todays-class-is-by-far-the-least-quantitative-but-it-will-give-you-the-flavor-of-what-well-do.}
\item ~
  \subsection{The math level should be accessible by junior/senior
  undergraduates and graduate students. For sophomores and frosh it
  probably gets increasingly hard to follow - although both of those
  class years have taken and passed the class. But there's probably more
  catch-up work if you're in an earlier
  class.}\label{the-math-level-should-be-accessible-by-juniorsenior-undergraduates-and-graduate-students.-for-sophomores-and-frosh-it-probably-gets-increasingly-hard-to-follow---although-both-of-those-class-years-have-taken-and-passed-the-class.-but-theres-probably-more-catch-up-work-if-youre-in-an-earlier-class.}
\item ~
  \subsection{Some people found handling the concepts challenging and
  others found them quite easy. Don't feel shy if you are challenged --
  just ask questions. No matter how smart you are, it's hard to keep up
  with live mathematical exposition: if I weren't the one giving the
  lecture, I would probably fall behind listening to it from time to
  time. So (as I said previously!) do everyone a favor and ask when
  you're getting
  behind.}\label{some-people-found-handling-the-concepts-challenging-and-others-found-them-quite-easy.-dont-feel-shy-if-you-are-challenged-just-ask-questions.-no-matter-how-smart-you-are-its-hard-to-keep-up-with-live-mathematical-exposition-if-i-werent-the-one-giving-the-lecture-i-would-probably-fall-behind-listening-to-it-from-time-to-time.-so-as-i-said-previously-do-everyone-a-favor-and-ask-when-youre-getting-behind.}
\end{itemize}

    \begin{itemize}
\item ~
  \section{0.4 Outline of class}\label{outline-of-class}
\item ~
  \subsection{This class is about financial risk management, with a tilt
  toward ``buy side'' risk
  management.}\label{this-class-is-about-financial-risk-management-with-a-tilt-toward-buy-side-risk-management.}
\item
  \#\#\# In the financial world, the ``buy side'' consists of
  institutions that invest money for clients. Examples are mutual funds,
  hedge funds, asset managers serving pension plans, college endowments,
  and sovereign wealth funds.
\item
  \#\#\# The ``sell side'' consists of broker-dealers who execute
  transactions; these are mainly divisions of large banks. Of course buy
  side institutions both buy and sell, and sell side institutions both
  buy and sell, so it's unclear why they have the names they have.
\item ~
  \subsection{Prerequisites: a solid undergraduate mathematics
  education. Knowledge of calculus, probability and statistics. If you
  have taken one or more of BEM 105, Ma 112, or ACM/ESE 118, you
  probably have the right
  background.}\label{prerequisites-a-solid-undergraduate-mathematics-education.-knowledge-of-calculus-probability-and-statistics.-if-you-have-taken-one-or-more-of-bem-105-ma-112-or-acmese-118-you-probably-have-the-right-background.}
\item ~
  \subsection{Lectures will be as
  follows:}\label{lectures-will-be-as-follows}
\item
  \#\#\# Lectures 1\&2: Risk, Uncertainty and Profit (Frank Knight,
  1921). Basic concepts of risk as uncertainty about the future. Basic
  economics. Utility theory.
\item
  \#\#\# Lectures 3\&4: Interest rate risk. Yield curves and yield curve
  dynamics. Principal components. Litterman-Scheinkman. Short rate
  models.
\item
  \#\#\# Lectures 5\&6: Coherent risk measures (Artzner, Delbain, Eber
  and Heath), Bayes Theorem. Markowitz efficient frontier; portfolio
  selection. Michaud -- resampled efficient frontier. Risk Parity.
  Black-Litterman.
\item
  \#\#\# Lectures 7\&8: Portfolio volatility models: APT, Factor, PCA.
  Normality \& non-normality. Generating portfolio distributions:
  Historical, Delta-normal,
\item
  \#\#\# Lectures 9\&10: Generating portfolio distributions, cont'd:
  Delta-gamma, Monte-Carlo. Scenario analysis and stress testing. Fat
  tails. Regime switching. Extreme value theory.
\item
  \#\#\# Lectures 11\&12: Time-series volatility modeling. Heston's
  model. GARCH and variants. Practical methods to predict volatility.
\item
  \#\#\# Lectures 13+14: Correlation measures. Copula functions.
  Anticipating correlations. MacGyver method (Engel).
\item
  \#\#\# Lectures 15\&16: Credit risk: Capital Structure. Structural
  models. Merton and KMV models. Credit convexity.
\item
  \#\#\# Lectures 17\&18: Credit: Reduced form models, credit default
  swaps, Li copula. Hedge fund risk management
\end{itemize}


    % Add a bibliography block to the postdoc
    
    
    
    \end{document}
